Descrição do projeto\+: ~\newline
O objetivo principal deste trabalho é fazer alterar o brilho de um led em dois modos\+: manual ou automático. ~\newline
Através do modo manual, o brilho do L\+ED é controlado manualmente através de dois dos botões da placa. Um para aumentar e outro para diminuir o brilho do L\+ED. ~\newline
Através do modo automático, é pedido ao utilizador que programe o brilho do L\+ED durante o dia, tendo este que introduzir os valores do brilho do L\+ED desejados para todas as horas (0 a 23h). Caso não seja introduzido nenhum valor numa determinada hora, o L\+ED ficará desligado. ~\newline
O controlo do brilho do L\+ED no modo automático é feito através de um controlador P. ~\newline
A tensão é obtida através da leitura da A\+DC, com um período de amostragem de 1 seg. ~\newline
A variação do brilho do led é feita através de um sinal P\+WM, ao variar o seu duty-\/cycle. ~\newline
A implementação deste programa foi feita através de threads, sendo a comunicação entre os processos feita através de F\+I\+F\+Os.\begin{DoxyAuthor}{Author}
Beatriz Carvas ~\newline
 Dário Fernandes ~\newline
 Guilherme Cajeira 
\end{DoxyAuthor}
